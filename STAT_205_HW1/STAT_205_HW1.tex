\documentclass[12pt]{article}  
% Required Packages 
\usepackage{amsmath}
\usepackage{multirow}
\usepackage{array}
\usepackage[noblocks]{authblk} 
\usepackage{graphicx}
\usepackage{color}
\renewcommand\arraystretch{1.5}

\title{STAT 205 Home Work 1}
\author{Honglei Ren}
\begin{document}
\maketitle

\subsection{Developmental Trajectory Clustering}
The developmental trajectories were hierarchically clustered based on their geometric distance in PCA space. More specifically, the \textit{fcluster} method in scikit-learn package was used in hierarchical clustering \textcolor{blue}{(citation?)}, and the geometric distance between trajectories A and B were defined as the sum of the pair-wised Euclidean distance between two corresponding stages, i.e. 


\begin{equation}
||A - B||_F = \sqrt{\sum_{i=1}^{m} \sum_{j=1}^{n} (A_{i, j}- B_{i,j})^2}
\label{eq:traj_dist}
\end{equation}

, where $||\cdot||_F$ is the Frobenius norm, $A$ and $B$ are two developmental trajectories represented by $m$ by $n$ matrices, $m$ is the number of developmental stages in single cell data, $n$ is the number of PCA components used in clustering. Each trajectory is a dimension reduced coexpression landscape of a gene pair, and all gene pairs in clustering were downloaded from the gene pair list proposed by \textcolor{blue}{(Briggs et al ref??)}

\end{document}
